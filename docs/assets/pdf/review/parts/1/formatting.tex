\chapterimage{chapter_head_2.pdf}

\chapter{Formatting}

\section{Size of a File}\index{Formatting!Size of a File}

File size should be no more than 500 lines of code \textbf{approximately}

\section{The Newspaper Metaphor}\index{Formatting!The Newspaper Metaphor}

A source file should be like a newspaper article, in which a headline tells you what the story is all about and as you do down you see more details.

A Java class name should be simple and explanatory. The name, by itself, should be sufficient to tell us whether we are in the right module or not.

The topmost parts of the source file should provide the high-level concepts and algorithms. Detail should increase as we move downward, until at the end
we find the lowest level functions and details in the source file.

\begin{marker}
Java Class name should be self-explanatory

The class doc should profide high-level concepts and algorithms.
\end{marker}

\section{Vertical Density}\index{Formatting!Vertical Density}

The vertical density implies close association. So lines of code that are tightly related should appear vertically dense. Listing~\ref{formating-vertical-density-bad} is an example of how too much comments screws up vertical density: 

\begin{tcolorbox}[breakable, colback=red!10!white, colframe=red!85!black, sidebyside, righthand width = 3cm, tikz lower, label=formating-vertical-density-bad]

\begin{lstlisting}[language = java, basicstyle=\small]
public class ReporterConfig {

    /**
     * The class name of the reporter listener
     */
    private String m_className;
    
    /**
     * The properties of the reporter listener
     */
    private List<Property> m_properties = new ArrayList<Property>();
    
    public void addProperty(Property property) {
        m_properties.add(property);
    }
    
    ...
}
\end{lstlisting}

\tcblower

\path[fill = yellow, draw = yellow!75!red] (0, 0) circle (1cm);
\fill[red] (45:5mm) circle (1mm);
\fill[red] (135:5mm) circle (1mm);
\draw[line width=1mm,red] (230:6mm) arc (145:35:5mm);

\end{tcolorbox}

\begin{marker}
Avoid doc on instance variables whenver you can
\end{marker}

\section{Dependent Functions}\index{Formatting!Dependent Functions}

If one function calls another, they should be vertically close, and the caller should be above the callee, if at all possible. This gives the program a natural flow. If the convention is followed reliably, readers will be able to trust that function definitions will follow shortly after their use.

\begin{marker}
If one function calls another, they should be vertically close, and the caller should be above the callee.
\end{marker}

\section{Conceptual Affinity}\index{Fomratting!Conceptual Affinity.}

Anything conceptually related should be vertically placed close to each other. For example, a group of functions perform a similar operation (share a common naming
scheme and perform variations of the same basic task)

\begin{marker}
Place conceptually related pieces close to each other vertically
\end{marker}

\section{Vertical Ordering}\index{Formatting!Vertical Ordering}

In general we want function call dependencies to point in the downward direction. we expect the most important concepts to come first, and we expect them to be expressed with the least amount of polluting detail. We expect the low-level details to come last. This allows us to skim source files, getting the gist from the first few functions, without having to immerse ourselves in the details.

\begin{marker}
Put the most important concept about a class on top.
\end{marker}
