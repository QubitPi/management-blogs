\chapterimage{chapter_head_2.pdf}

\chapter{Keep Entropy in Mind while Reviewing}

\section{What is Software Entropy?}

While software development is immune from almost all physical laws, entropy hits us hard. Entropy is a term from physics that refers to the amount of “disorder” in a system. Unfortunately, the laws of thermodynamics guarantee that the entropy in the universe tends toward a maximum. When disorder increases in software, programmers call it “software rot.”

There are many factors that can contribute to software rot. The most important one seems to be the psychology, or culture, at work on a project. Even if you are a team of one, your project's psychology can be a very delicate thing. Despite the best laid plans and the best people, a project can still experience ruin and decay during its lifetime. Yet there are other projects that, despite enormous difficulties and constant setbacks, successfully fight nature's tendency toward disorder and manage to come out pretty well.

What makes the difference? The broken-window theory explains it.

\begin{marker}
Spot "Broken Windows" and ask code author to fix it right away.
\end{marker}
