\chapterimage{chapter_head_2.pdf}

\chapter{Size of Pull Request}

\section{How Big is Appropriate?}

The story of Stone Soup and Boiled Frogs(https://en.wikipedia.org/wiki/Stone\_Soup) told us one thing about dealing with politics of software developments:

You may be in a situation where you know exactly what needs doing and how to do it. The entire system just appears before your eyes—you know it’s right. But ask permission to tackle the whole thing and you’ll be met with delays and blank stares. People will form committees, budgets will need approval, and things will get complicated. Everyone will guard their own resources. Sometimes this is called “start-up fatigue.”

It’s time to bring out the stones. Work out what you can reasonably ask for. Develop it well. Once you’ve got it, show people, and let them marvel. Then say “of course, it would be better if we added .” Pretend it’s not important. Sit back and wait for them to start asking you to add the functionality you originally wanted. People find it easier to join an ongoing success. Show them a glimpse of the future and you’ll get them to rally around

Code reviewers should judge the size of a PR using SS story as a guidance, is this change small enough to start something which can be shown to people? If it's too big, as code author to break it into multiple PR's

\begin{marker}
Be a Catalyst for Change
\end{marker}

Keeping the PR size small has another advantage. Consider the other side. The villagers think about the stones and forget about the rest of the world. We all fall for it, every day. Things just creep up on us.

We’ve all seen the symptoms. Projects slowly and inexorably get totally out of hand. Most software disasters start out too small to notice, and most project overruns happen a day at a time. Systems drift from their specifications feature by feature, while patch after patch gets added to a piece of code until there’s nothing of the original left. It’s often the accumulation of small things that breaks morale and teams.

\begin{marker}
As a code reviewer, always remember the Big Picture
\end{marker}
