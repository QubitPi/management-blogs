\chapterimage{chapter_head_2.pdf}

\chapter{Introduction}

\section{KEEP it Clean}

It's not enough to WRITE the code well. The code has to be kept clean over time. We've all seen code rot and degrade as time passes. So we must take an active role in preventing this degradation.

The Boy Scouts of America have a simple rule that we can apply to our profession.

\begin{tcolorbox}[colback=green!5!white,colframe=green!75!black]
Leave the campground cleaner than you found it.
\end{tcolorbox}

If we all checked-in our code a little cleaner than when we checked it out, the code
simply could not rot. The cleanup doesn't have to be something big. Change one variable name for the better, break up one function that's a little too large, eliminate one small bit of duplication, clean up one composite \ubuntubox{\lstinline{if}} statement

\textbf{Whenever possible, ask PR authors to do self-review as much as possible for the purpose stated above}
