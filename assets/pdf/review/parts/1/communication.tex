\chapterimage{chapter_head_2.pdf}

\chapter{Communication}

\section{Know What You Want to Say}

Probably the most difficult part of the more formal styles of code review comment used in business is working out exactly what it is you want to say.

Plan what you want to say. Write an outline. Then ask yourself, “Does this get across whatever I'm trying to say?” Refine it until it does. Jot down the ideas you want to communicate, and plan a couple of strategies for getting them across.

\section{Know the Code Author}

You're communicating only if you're conveying information(code changes). The following technique helps you to structure your code review commetns:

\begin{tcolorbox}[
    title = The \textbf{WISDOM} acrostic - understanding the audience(code author)
]

\textbf{W}hat do you want them to learn?

What is their \textbf{i}nterest in what you've got to say?

How \textbf{s}ophisticated are they?

How much \textbf{d}etail do they want?

Whom do you want to \textbf{o}wn the information?

How can you \textbf{m}otivate them to listen to you?

\end{tcolorbox}

\section{Choose a Style}

Adjust the style of your delivery to suit your audience. Some people want a formal “just the facts” briefing. Others like a long, wide-ranging chat before getting down to business. When it comes to written documents, some like to receive large bound reports, while others expect a simple memo or e-mail. If in doubt, ask.

Thinking about the other side's style is important, keeping you, one side of communication as well, clear is also important.

\section{Make It Look Good}

Your ideas are important. They deserve a good-looking vehicle to convey them to your audience.

Too many developers (i.e. code reviewers) concentrate solely on content when producing written code review comments. We think this is a mistake. Any chef will tell you that you can slave in the kitchen for hours only to ruin your efforts with poor presentation.

There is no excuse today for producing poor-looking code reviews. Modern word processors (along with layout systems such as markdown) can produce stunning output. You need to learn just a few basic rules.

Making a good-looking review encourages not only you, but also the other side to read it with enjoyable experiences.

\section{Be a Listener}

There's one technique that you must use if you want people to listen to you: listen to them. Even if this is a situation where you have all the information, even if this is a formal meeting with you standing in front of 20 suits—if you don't listen to them, they won't listen to you.

Encourage people to talk by asking questions. Turn the meeting into a dialog, and you'll make your point more effectively. Who knows, you might even learn something.

