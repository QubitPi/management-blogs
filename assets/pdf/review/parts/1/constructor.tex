\chapterimage{chapter_head_2.pdf}

\chapter{Constructors}

\section{Constructors in general}

A  conventional distinction is made in Java between the "default constructor" and a "no-argument constructor":

\begin{definition}[Default constructor]
The default constructor\index{Constructors!default constructor} is the constructor provided by the system in the absence of any constructor provided by the programmer. Once a programmer supplies any constructor whatsoever, the default constructor is no longer supplied.
\end{definition}

\begin{definition}[No-args constructor]
A no-argument constructor\index{Constructors!no-argument constructor}, on the other hand, is a constructor provided by the programmer which takes no arguments. This distinction is necessary because the behavior of the two kinds of constructor are unrelated: a default constructor has a fixed behavior defined by Java, while the behavior of a no-argument constructor is defined by the application programmer.
\end{definition}

\begin{marker}
    It is important that JavaDoc comments makes clear distinctions between the two types of constructors. Watch out for any discrepancies in author's code.
\end{marker}

\section{Do Not Pass 'this' out of a Constructor}\index{Constructors!Do Not Pass 'this' out of a Constructor}

Within a class, the \mintinline{java}{this} Java keyword refers to the native object, the current instance of the class. Within a constructor, you can use the keyword in 3 different ways:

\begin{itemize}
    \item on the first line, you can call another constructor, using \mintinline{java}{this(...);}
    \item as a qualifier when referencing fields, as in \mintinline{java}{this.name;}
    \item as a reference passed to a method of some other object, as in \mintinline{java}{blah.operation(this);}
\end{itemize}

You can get into trouble with the last form. The problem is that, inside a constructor, the object is not yet fully constructed.

\begin{remark}
    An object is only fully constructed after its constructor completely returns, and not before
\end{remark}

But when passed as a parameter to a method of some other object, the this reference should always refer to a fully-formed object. This problem occurs mainly with listeners. Here's an example which illustrates the point:

\begin{minted}{java}
import java.util.Observable;
import java.util.Observer;

public final class EscapingThisReference {

    // A RadioStation is observed by the people listening to it.
    static final class RadioStation extends Observable {
        // elided
    }

    /*
    * A listener which waits until this object is fully-formed before it lets it be referenced by the outside world.
    * Uses a private constructor to first build the object, and then configures the fully-formed object as a listener.
    */
    static final class GoodListener implements Observer {

        /** Factory method. */
        static GoodListener buildListener(String personsName, RadioStation station) {
            //first call the private constructor
            GoodListener result = new GoodListener(personsName);
            //the 'result' object is now fully constructed, and can now be
            //passed safely to the outside world
            station.addObserver(result);
            return result;
        }

        @Override
        public void update(Observable station, Object data) {
            //..elided
        }

        private String personsName;

        /** Private constructor. */
        private GoodListener(String personsName) {
            this.personsName = personsName; //ok
        }
    }
    /**
    * A listener which incorrectly passes a 'this' reference to the outside world before construction is completed.
    */
    static final class BadListenerExplicit implements Observer {

        /** Ordinary constructor. */
        BadListenerExplicit(String personsName, RadioStation station) {
            this.personsName = personsName; //OK
            //DANGEROUS - the 'this' reference shouldn't be passed to the listener,
            //since the constructor hasn't yet completed; it doesn't matter if
            //this is the last statement in the constructor!
            station.addObserver(this);
        }

        @Override
        public void update(Observable station, Object data) {
        //..elided
        }

        private String personsName;
    }
    /**
    * Another listener that passes out a 'this' reference to the outside  world before construction is completed; here,
    * the 'this' reference is implicit, via the anonymous inner class.
    */
    static final class BadListenerImplicit {

        /** Ordinary constructor. */
        BadListenerImplicit(String personsName, RadioStation station) {
            this.personsName = personsName; //OK
            //DANGEROUS
            station.addObserver(
                new Observer() {
                    @Override
                    public void update(Observable observable, Object data) {
                        doSomethingUseful();
                    }
                });
        }

        private void doSomethingUseful() {
            //..elided
        }

        private String personsName;
    }
}
\end{minted}

\begin{marker}
    Find and tell code authors to avoid passing \mintinline{java}{this} out of constructor 
\end{marker}

\section{Avoid JavaBeans Style of Construction}\index{Constructors!Avoid JavaBeans Style of Construction}

Constructing a Model Object in many steps, first by calling a no-argument constructor, and then by calling a series of setXXX methods, is something to be avoided, if possible, because:

\begin{itemize}
    \item it is more complex, since one call is replaced with many
    \item there is no simple way to ensure that all necessary setXXX methods are called
    \item there is no simple way to implement a class invariant
    \item it allows the object to be in intermediate states which are not immediately useful, and perhaps even invalid
    \item it is not compatible with the (very useful) immutable object pattern
    \item it does not follow the style of many design patterns, in which concrete classes usually have arguments passed to their constructor. This is a recurring theme in the Design Patterns book. Constructors usually act as a "back door" for data which is needed by an implementation, but which cannot appear in the signature of some higher-level abstraction (since it would pollute such an abstraction with data specific to an implementation)
\end{itemize}

Of course, JavaBeans follow this pattern of "no-argument constructor plus setXXX". However, JavaBeans were originally intended for a very narrow problem domain - manipulation of graphical components in an IDE. As a model for the construction of typical data-centric objects (Model Objects), JavaBeans seem completely inappropriate. One can even argue strongly that, for Model Objects, the JavaBeans style is an anti-pattern, and should be avoided unless absolutely necessary.

Reading \href{https://www.oracle.com/technetwork/java/javase/documentation/spec-136004.html}{JavaBeans Spec} helps you completely understand why JavaBean is not a good idea in this context. It is important to understand Java Beans in terms of how Java is used today. The bean initiates itself as a GUI concept. Java today refers to "bean" in a completely different context defined by its popularities in non-GUI developments.

\begin{definition}{What is a Bean (Historically)?}
A Java Bean is a reusable software component that can be manipulated visually in a builder tool.
\end{definition}

Simply put, this is defined in the old Java Applet in which you click through to configure the applet before you hit run. The bean is used to record what you clicked and later spins up the program accordingly.

But today you do not hear Java Applet anymore, but you still hear about beans. Why? Because bean is usable to other popular contexts. One of them is JPA (or Persistent Storage)

This is very clear and simple. No mention is made here, or anywhere else in the specification, of Model Objects used to encapsulate database records. The whole idea of a Model Object is entirely absent from the Java Bean specification, and for good reason: the manipulation of graphical controls and the representation of database records(See 2.1.1 Beans v. Class Libraries in \href{https://www.oracle.com/technetwork/java/javase/documentation/spec-136004.html}{JavaBeans Spec}) are unrelated issues.

The life cycle of a Java Bean, starting with its no-argument constructor, is derived mainly from the task of configuring a graphical control having some default initial state. For the intended domain, this is indeed a reasonable design. For Model Objects, however, the idea of configuration in this sense is entirely meaningless.

The role of a Model Object is to carry data. Since a no-argument constructor can only build an object that is initially empty, then it cannot contribute directly to this aim. It does not form a natural candidate for the design of a Model Object.

To put data into empty Java Beans, frameworks typically call various setXXX methods using reflection. There is nothing to prevent them from calling constructors in exactly the same way.

The widespread adoption of Java Beans as a supposedly appropriate design for Model Objects seems to be an error.

\begin{marker}
    Unless absolutely necessary, do not set through objects.
\end{marker}

\section{Initializing fields to 0, false, or null is redundant}\index{Constructors!Initializing fields to 0, false, or null is redundant}

One of the most fundamental aspects of a programming language is how it initializes data. For Java, this is defined explicitly in the language specification(\lstinline{https://docs.oracle.com/javase/specs/jls/se8/html/jls-4.html#jls-4.12.5}). For fields and array components, when items are created, they are automatically set to the following default values by the system:

\begin{itemize}
    \item numbers: 0 or 0.0
    \item booleans: false
    \item object references: null
\end{itemize}

This means that explicitly setting fields to 0, false, or null (as the case may be) is unnecessary and redundant. Since this language feature was included in order to, in part, reduce repetitive coding, it's a good idea to take full advantage of it. Insisting that fields should be explicitly initialized to 0, false, or null is an idiom which is likely inappropriate to the Java programming language

Furthermore, setting a field explicitly to 0, false, or null may even cause the same operation to be performed twice (depending on your compiler). For example:

\begin{minted}{java}
public final class Quark {

    public Quark(String aName, double aMass){
        fName = aName;
        fMass = aMass;
    }

    //PRIVATE

    //WITHOUT redundant initialization to default values
    //private String fName;
    //private double fMass;

    //WITH redundant initialization to default values
    private String fName = null;
    private double fMass = 0.0d;
}
\end{minted}

If the bytecode of the Quark class is examined, the duplicated operations become clear (here, Oracle's javac compiler was used):


\begin{tcolorbox}[
skin=bicolor,colframe=blue!70!black,fonttitle=\bfseries,
    colback=blue!20!white,colbacklower=green!20!white,left=0ex,right=0ex,
    sidebyside,
  title={Title in the first language},
  after title={\hfill Title in the second language},
  width=\linewidth*1
  ]
    \begin{lstlisting}
    >javap -c -classpath . Quark
Compiled from Quark.java
public final class Quark extends java.lang.Object {
public Quark(java.lang.String,double);
}

Method Quark(java.lang.String,double)
0 aload_0
1 invokespecial #1 <Method java.lang.Object()>
4 aload_0
5 aload_1
6 putfield #2 <Field java.lang.String fName>
9 aload_0
10 dload_2
11 putfield #3 <Field double fMass>
14 return
    \end{lstlisting}
    \tcblower

    \begin{lstlisting}
    >javap -c -classpath . Quark
Compiled from Quark.java
public final class Quark extends java.lang.Object {
public Quark(java.lang.String,double);
}

Method Quark(java.lang.String,double)
0 aload_0
1 invokespecial #1 <Method java.lang.Object()>
4 aload_0
5 aconst_null
6 putfield #2 <Field java.lang.String fName>
9 aload_0
10 dconst_0
11 putfield #3 <Field double fMass>
14 aload_0
15 aload_1
16 putfield #2 <Field java.lang.String fName>
19 aload_0
20 dload_2
21 putfield #3 <Field double fMass>
24 return
    \end{lstlisting}
\end{tcolorbox}

\section{Constructors Shouldn't Start Threads}\index{Constructors!Constructors Shouldn't Start Threads}

There are two problems with starting a thread in a constructor (or static initializer):

\begin{enumerate}
    \item in a non-final class, it increases the danger of problems with subclasses
    \item it opens the door for allowing the this reference to be passed to another object before the object is fully constructed
\end{enumerate}
    

There's nothing wrong with creating a thread object in a constructor or static initializer - just don't start it there.